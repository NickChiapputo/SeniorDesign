\documentclass[conference]{IEEEtran}

\usepackage{cite}				% Used to cite references and figures
\usepackage{titling}			% Used to move title up when not in IEEEtran
\usepackage{subcaption}			% Used to add captions to subfigures
\usepackage{graphicx}			% Used to import pdf images
\usepackage{float}				% Used to keep images where they are defined using [H] tag
\usepackage{amsmath}			% Used for equation reference
\usepackage{amsfonts}			% Used for equation symbols
\usepackage[export]{adjustbox}	% Used to left align selected figures
%\usepackage{siunitx} 			% Used to align decimals in table
\usepackage{hyperref}
\usepackage{bookmark}
\usepackage{algorithm,algpseudocode}			% Used to create an algorithm block
% \usepackage{algorithmic}			% Used to create an algorithm block

% Adjust margins with geometry package
\usepackage[top=1in,bottom=1in,left=1in,right=1in]{geometry}

% Move title up
%\setlength{\droptitle}{-12.5em}

\begin{document}

	\title{ Product Title }

	\author{	Senior Design I Student Proposal \\
				University of North Texas Department of Electrical Engineering \\
				EENG 4910.001 \\ \\
				Nicholas Chiapputo, Brandon Jones, Tim McCoig, Samuel Simmons \\
				Faculty Advisor: Colleen Bailey, PhD
	}

	\maketitle

	% \begin{abstract}
	% \end{abstract}

	\section{Problem Definition}
		Drone with deliverable qualities. Request options are still undetermined. There is always a need for more drones to do things to eliminate human error.

		We are trying to make more jobs for drones to eliminate human error which hopefully will increase efficiency. This will create more time for brainstorming and idea creation to achieve more tasks per year. The specificity for the drone usage in our project is to create a path or job for a drone to either deliver objects/packages and or function as a vacuum/dusting service.

		The closest implemented project for the design mentioned is Amazon’s Prime delivery drone. The project is new and is being implemented either now or soon.

	\section{Researching and Generating Ideas}
		We need to research the basic knowledge on the design of our project with concepts and procedures to help define the rules for construction of our project. We created a criteria that will define the basic performance needs of our drone (delivery service). A list of constraints and standards will help with implementation of the project.

	\section{Conception, Requirements, and Specifications}
		The design conception is still under discussion. The schematic is in the works and should be a diagram of well thought out drone parts and what each section/part will function as or its job will be. Once we decide on a parts list and a time restraint on building/constructing drone to test/program, then our preliminary design will become complete.
	
	\section{Constraints}
		The list of constraints should be a large list. We need to discuss among the group on how we will implement and hurdle some of these constraints with a potential budget.

	\section{Ethical, Professional, and Contemporary Issues}
		This a potential section that is on its own and not listed within the requirements of the proposal but is definitely a discussion that must be considered. These issues may be a restraint and or a part of the standards conversation that will help us implement our projects design with zero road blocks.

	\section{Engineering Standards}
		Another section that is a part of the problem definition that will be brought out to be its own section. Standards could become a 

	\section{Design}


	\section{Parts List}


	\section{Prototype and System Integration}


	\section{Delivrables}


	\section{Timeline}


	\begin{thebibliography}{00}
		\bibitem{githubSource} [Online].Available: \url{https://github.com/NickChiapputo/SeniorDesign}
	\end{thebibliography}
\end{document}